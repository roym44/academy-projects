\documentclass{article}

%general packages:
\usepackage[utf8]{inputenc}		%use Unicode
\usepackage{tipa} 				%for IPA
\usepackage{tikz}				%for Diagrams
	\usetikzlibrary{positioning}		%for relative positioning 
\usepackage{graphicx}			%for inserting images
	\graphicspath{ {./} }			%the path for the images
\usepackage{hyperref}			%for links

%packages for Dynamic OT:
\usepackage{pifont}			%for pointing hand
\usepackage{arydshln}			%for dashed lines
\usepackage{rotating}			%for angled text


\usepackage{multirow} % for tables
\usepackage{enumitem} % for ordered list a-b-c 

\usepackage{fancyhdr}
\pagestyle{fancy}

\begin{document}	

\begin{titlepage}
    \begin{center}
        \vspace*{2cm}
        \Huge
        \textbf{Optional Schwa Deletion\\in French}
            
        \vspace{1.5cm}
        \LARGE
        \textbf{Roy Mayan}\\
        Submitted to: Dr. Ezer Rasin 
            
        \vfill
            
        Presented as a final paper of\\
        Advanced Phonology: Opacity and\\Phonological Architecture
            
        \vspace{0.8cm}
            
        \Large
        Linguistics Department\\
        Tel Aviv University\\
        February 2023
    \end{center}
\end{titlepage}

I wish to thank the people who participated in my study,  Clémence who helped my with the execution of the experiment, and my dear friends Hila and Maya for their input and proofing of the paper. 

\section*{Abstract}
{\sffamily\small
In French, schwa can be optionally deleted in various places of a given phrase. According to a rule-based analysis of a specific environment in which this occurs done by Dell (1973),  when several successive syllables contain schwas likely to be erased, a speaker can decide for each schwa instance whether to delete it or not. That behaviour leads us to have 4 different pronunciation options for a sentence with 2 schwas, with the exception of the option in which we delete both schwas - marked by Dell as incorrect. In this paper, I conducted an experiment in which French native speakers use their intuition to determine which options are valid. The purpose of the experiment was to test whether Dell's suggestions are characteristic of speakers today. This is directly related to the phenomenon's historical significance in the discussion over different theories of phonology. The results show strong correlation with Dell's work, yet I suggest a more thorough study for future work.}

\section{Introduction} % 1 - Introduction
Schwa deletion in French is a phonological pattern that has been extensively researched in phonological literature. A specific instance of such deletion was presented by Dell (1973) in his famous book "Les règles et les sons" (The rules and sounds) under "Schwas en syllabes contiguës" (Schwas in contiguous syllables) among many other cases of schwa deletion. This specific phenomenon is known as Optional Schwa Deletion (henceforth OSD), and has been examined several times later on,  challenging various phonological theories. A question that arises is whether native speakers of French approve/disprove of the different options of deleting the schwa in some cases as Dell suggested.\\

The goal of this paper is to try to answer this question by conducting an experiment, in order to determine whether Dell's findings are characteristic of  French speakers today. In section 2, I introduce the theoretical background for my research (Phonology and OSD in particular),  then I present my question and explain how the paper will try to address it. Then,  in section 3,  I present the experiment designed and run to test this question. Finally,  in section 4,  I summarize the conclusions that I draw from my research.

The data collected for this study is based on native French speakers. Both the recordings I used in my online study were made by a French speaker, and the people who participated in it and gave their native judgments are French.

\clearpage
\section{Background} % 2 - Background
In linguistics, the study of languages, phonology is the study of their pronunciation,  the sound system of a language and its properties.There are two common ways to describe a phonological system. One way uses rules (Rule-Based Theory) and another uses a system of ranked-constraints (Optimality Theory). Both theories assume that lexical items (words) have underlying representations and surface representations. The underlying representations are the
way words are encoded in the human brain, which is sometimes different than the way they appear on the surface (i.e. surface representations). The surface representations are the way words sound after applying rules or constraints according to the chosen phonological theory. 

\subsection{Terminology and basic definitions} % UG,  rule-based, OT, Opacity
\begin{enumerate}

  \item \textbf{Universal Grammar (UG)}: The premise of Generative Linguistics is the fact that native speakers have knowledge about
their language that allows them to produce and comprehend words and sentences that they have never heard before. The description of that knowledge is called a Grammar, and it is what humans are assumed to acquire when acquiring language. This led to the thought that humans are born with some characterization of the set of grammars that they are able to acquire,  which Chomsky proposed to call UG.
  \item \textbf{Serial Rule-Based Theory}: In this paper I focus on Dell's treatment of OSD which has been done using the rule-based theory. This theory was proposed by Chomsky \& Halle (1968),  and it is a rule-based theory of UG for the phonological module.  According to this theory, the phonology acquired by humans is formed by a lexicon and a set of ordered, language-specific rules that apply sequentially. The basic components of the theory are (informal,  simplified):
\begin{enumerate}
  \item A grammar consists of: A lexicon (a set of strings), and an ordered list of rules of the form A → B / C \_ D. \\ The primitive symbols that can be used in rule descriptions are: +, -, →,
round, high, voiced, …
 \item A grammar is interpreted as follows: The grammar takes an element from the lexicon and generates an output by
applying the rules to that element in a sequence.\\
To apply a rule A → B / C \_ D to an input, change every instance of A between C and D in the input to have the features in B.
\end{enumerate}

  \item \textbf{Optimality Theory (OT)}: OT is another theory of UG, presented by Prince \& Smolensky (1993), suggesting that surface representations arise from the optimal satisfaction of conflicting constraints. According to this theory, when humans acquire a language, they are learning the language’s rank of constraints. The basic components of the theory are (informal,  simplified):
\begin{enumerate}
  \item A generator that for every input creates the list of possible candidates (outputs).
  \item Constraints that are described as a set of rules that affect the choice for the optimal candidate. For every language, these constraints are strictly ranked and violable. There are two fundamental classes of constraints: Faithfulness constraints that aspire to preserve the outputs as close as possible to the underlying representations, and Markedness constraints that impose requirements on the structural well-formedness of the output. The two classes of constraints can contradict each other. 
  \item An evaluator that chooses the optimal candidate based on the constraints and their ranks. The selected candidate is the output. The optimal candidate is the candidate that either violates no constraints or violates only the least significant ones.
\end{enumerate}

  \item \textbf{Opacity}: Some rules and generalizations in phonology fall short of categorial truth, and in some cases the expected result they state is hidden by other aspects of the phonological system. When a generalization is partially obscured in this way,  it is said to be \textit{opaque}.  Opaque generalizations - generalizations that lose support on the surface - have played an important role in the development of phonological theory since the 1950’s and remain at the center of debate to this day.  Dealing with Dell's analysis using the rule-based theory, I will present, in short, the definitions given by Kiparsky (1971) regarding this issue of opacity and how it is analyzed in rule-based phonology:\\
A rule P of the form A → B / C \_ D is opaque if there are surface representations with:
\begin{enumerate}
  \item Instances of A in the environment C \_ D.
  \item Instances of B derived by P in environments other then C \_ D.
\end{enumerate}
Moreover, rules may have different functional relationships to one another. In the least interesting case, a pair of rules may not interact at all, but when they do interact, the relationship between them can often be classified as \textit{feeding} or \textit{bleeding}. \\
Given two rules P, Q such that P precedes Q in a given derivation:
\begin{enumerate}
  \item P feeds Q: P creates additional inputs to Q.
  \item P bleeds Q: P removes potential inputs to Q.
  \item Q counterfeeds P: Q would have created additional inputs to P if Q were to apply before P.  This interaction can cause an \textit{underapplication} opacity, on the surface it seems that P didn't apply, though there is a valid environment for it to do so.  The rule fails to apply when the environment is met.
  \item Q counterbleeds P: Q would have removed potential inputs to P if Q were to apply before P. This interaction can cause an \textit{overapplication} opacity, on the surface it seems that P applied, though there is not a valid environment for it to do so. The environment is not met on the surface.
\end{enumerate}

  \item Other definitions: Dell uses the sign \# to mark the boundaries of a word, and * for incorrect derivations.

\end{enumerate}

\subsection{OSD in French}
% intro
The vowel transcribed as [ə] in French is a mid front rounded vowel whose specific realization varies across contexts and dialects.  It has been the subject of a large body of research,  because (among other reasons) it is optional in a variety of positions.  In this paper I discuss a specific context that Dell suggested in 1973 in his book translated as \textit{Generative phonology and French phonology}.  A big part of that book is devoted to a detailed discussion of the behaviour of schwa. Among many cases of schwa deletion, Dell presented this case of OSD as follows, starting with the following rule:\\

% present the rule, and examples
\begin{center}
\textbf{VCE (optional): ə → ø / V\#C\_} \\
(meaning, a schwa is deleted after a vowel, word boundary and a consonant)
\end{center}
When several successive syllables contain schwas likely to be erased by VCE, a speaker speaking at a normal rate tends to drop the maximum number of schwas possible.  This is a trend rather than an absolute necessity. \\ 
Let's consider \textit{(tu as) envie de te battre} "(you) feel like fighting",  represented as [\textipa{\~avi\*\#d@\*\#t@\*\#batr}].
According to Dell, we can decide for each environment whether to delete the schwa or not, and that's how we get the following options:
\begin{enumerate}
  \item Output is [\textipa{\~avi\*\#d@\*\#t@\*\#batr}] - no deletion
  \item Output is [\textipa{\~avi\*\#d\*\#t@\*\#batr}] - the first schwa is deleted, but not the second 
  \item Output is [\textipa{\~avi\*\#d@\*\#t\*\#batr}] - the second schwa is deleted, but not the first 
  \item Output is *[\textipa{\~avi\*\#d\*\#t\*\#batr}] - deleting both schwas is not possible
\end{enumerate}
% self-bleeding, left-to-right
We will analyze the application of the rule left-to-right:
\begin{itemize}
  \item Given the basic form [\textipa{\~avi\*\#d@\*\#t@\*\#batr}],  we can first decide to delete the first schwa or not, because the environment is met.
  \item If we apply it, we get option 2 - and then the environment for the second schwa is not met and we have no more options.
  \item If not, we continue to the second schwa and we can again decide whether to delete or not.
  \item If we apply the rule, we get option 3, otherwise we stay with option 1 (the original one).
  \item We don't generate option 4 - and that is the wanted situation.
\end{itemize}
The reason that this left-to-right operation works, is the fact that if we deleted the first schwa, we would prevent the deletion of the second schwa (otherwise that would lead us to option 4 which is wrong). This is a case of \textit{self-bleeding}.
In addition, if we were to apply the rule simultaneously,  we would get only option 4 and not the other three correct options.  Also, right-to-left application would generate the wrong output, leading us to option 4. \\

Another example provided by Dell: \textit{tu le retrouves} "you find it", represented as [\textipa{t\"u\*\#l@\*\#r@\*\#truv}]. The options are:
\begin{enumerate}
  \item Output is [\textipa{t\"u\*\#l@\*\#r@\*\#truv}]
  \item Output is [\textipa{t\"u\*\#l\*\#r@\*\#truv}]
  \item Output is [\textipa{t\"u\*\#l@\*\#r\*\#truv}]
  \item Output is *[\textipa{t\"u\*\#l\*\#r\*\#truv}]
\end{enumerate}
% direction of the rule, [kstr] example!
Now,  let us pay special attention to the issue of the direction in which the rule applies, and the consonant sequences it generates.
We saw that left-to-right application generates the correct derivation, and if we apply VCE right-to-left we will get the wrong derivation (option 4). \\
Dell gives another example with the phrase: \textit{il veut que ce travail soit bien fait} "he wants this job done well".  We focus on the sequence containing the schwas which is [\textipa{v\"o\*\#k@\*\#s@\*\#tra}]. The options are:
\begin{enumerate}
  \item Output is [\textipa{v\"o\*\#k@\*\#s@\*\#tra}]
  \item Output is [\textipa{v\"o\*\#k\*\#s@\*\#tra}]
  \item Output is [\textipa{v\"o\*\#k@\*\#s\*\#tra}]
  \item Output is *[\textipa{v\"o\*\#k\*\#s\*\#tra}]
\end{enumerate}
Given that option 4 is incorrect, one can argue that perhaps the rule does apply right-to-left, and the fact that this option is marked wrong, is due to the difficulty of the pronunciation of the sequence generated [kstr].  Here Dell presents the following fact: banning certain consonant clusters in the VCE output cannot be attributed to a general constraint which would prohibit certain sequences to appear in phonetic representations. For example, option 4 is invalid although the group [kstr] can be found in the phonetic representation of the word \textit{extraordinaire} [extraordinary]. Dell claims that the well or badly formed character of the three consonant cluster in the output of VCE does not depend on the characteristics of the consonants of which this cluster is made, but on the way in which it was created by VCE.  In short, [kstr] can be found as a valid sequence in French, there is no technical/articulatory difficulty to pronounce it.

If we didn't have this example that tells us that [kstr] is a valid sequence in the language, we could argue for an alternative explanation: The rule is applying right-to-left (with self-counterbleeding), so the phonology does generate all 4 options. Then, there is an external constraint that prevents pronouncing [kstr] (due to articulatory reasons) and that's why speakers count that option as wrong. However, if we see that indeed valid sequences in the language such as [kstr], when generated using the rule, are judged as wrong - we would favor the explanation of a left-to-right application.  

\subsection{OSD's Significance} % Iterativity, local optionality
As mentioned, OSD has several interesting attributes that gained attention in literature.  I would like to focus on two important appearences of OSD in phonological history: the first is related to Iterative rule application (as an argument for self-bleeding),  and the second is in the field of OT and how it deals with local optionality. I will give a brief explanation on each context in which OSD was mentioned, and focus on its properties that made it relevant to those contexts.\\

\begin{description}
   \item[Iterativity] This example from French was used as an argument against a theory of Chomsky \& Halle (1968) which declared that rules operate non-iteratively: that is, if a specific rule has several environments in which it can apply, it would apply to all of them together - simultaneously.  The prediction of a non-iterative application is no self-feeding/self-bleeding interactions. The reason OSD is an evidence against their theory, is due to the fact that here there seems to be an interaction of \textit{self-bleeding}: if the rule applies in one position, it ruins its potential environment to apply in a second position. If the rule were to apply simultaneously on both environments in the input, it would apply twice - and that is not what actually happens (according to Dell). It appears to be that if there are several environments, the deletion must take place one after another, so that the rule can bleed itself and generate the correct output.

   \item[Local Optionality] The choice itself whether to delete the schwa seems to be local: that is, if there is a word/phrase with several potential positions for the deletion, then we can decide seperately for each position whether to delete or not. This attribute of locality challenged OT.
If we look at locality in the rule-based theory, it is simple to explain - we can mark the rule as "optional" meaning it can apply or not. On the other hand, in OT we do not deal with bare rules/processes as a grammatical object that we can mark as optional or not. One approach to deal with this issue was using unnecessary ranking between different constraints, so an optional phonological process can be represented as a "tie" between a faithfulness and a markedness constraint.  OSD also challenges this because the optionality in OT is still global - either constraint can be ranked higher, and we receive only two possible options for the application of the process. The main issue here is that OT fails in capturing the locality of the decisions, which is position-specific. Therefore , Riggle \& Wilson (2005) offered an extended version for OT that can generate local optionality: "We argue that the current dominant approach to optionality in OT,  which claims that multiple outputs arise from reranking of standard OT constraints,  is not descriptively adequate... The central idea of our proposal is that... standard constraints are replaced with position-specific versions."
\end{description}

\subsection{Research Question}
As previously stated, the data regarding OSD which was cited many times later, comes from Dell's work. As of now, we do not know exactly to what extent his generalisations are correct, and whether French speakers do prefer these patterns in different combinations of words.

My main research question is whether speakers of French are willing to accept the patterns Dell suggested in his analysis of OSD back in 1973.  Do they prefer the opaque (caused by self-counterbleeding) or transparent (caused by self-bleeding) forms?

Another question is related to the direction in which the rule applies. As mentioned,  Dell's [kstr] example does favor left-to-right, but is this what happens in most cases? Are there any counter-examples?\\

The importance of knowing whether Dell's suggestions are characteristic of speakers today is directly related to its historical significance in the discussion over different theories of phonology,  as explained above (the connection between iterativity and opacity, locality, optionality) - that is the motivation for this study. I will try to address this question using an experiment, as described in detail in the following section.


\clearpage
\section{Method} % 3 - Main
In order to test native French speakers' knowledge of their language to verify Dell's analysis, they were exposed to recordings of different pronunciation options and asked to rate them.  For online data collection,  the software PsyToolkit was used (Stoet, 2010, 2017). After constructing the basic phrases myself, I consulted a French speaker who also recorded in her voice all the different versions as described below. 


\subsection{Design}
While designing the experiment, my goal was to make it as simple and concise as possible, to obtain more engagement from people completing it fully. Therefore, I decided to define the following conditions:
\begin{itemize}
  \item 4 main phrases.
  \item About 3-5 different pronunciations of each.
  \item The phrases contain only 2 schwas.
\end{itemize}
% Phrases (a)-(b)
First,  I constructed 2 phrases based on the [kstr] case. The first one is the original phrase presented by Dell - \textit{il veut que ce travail soit bien fait}, where \textit{que} and \textit{ce} are the source for the schwas, leading to the sequence: [\textipa{v\"o\*\#k@\*\#s@\*\#tra}].
The second phrase is based on the same principle but with a different sequence: [kspr]. Using a French corpus I found online (\cite{corpus}), I was searching for similar sequences to [kstr] (that appears in the word \textit{extraordinare}.) I came up with [dstr] as an option, but could not find any French words containing it. I found the sequence [kspr] (orthographically "xpr") such as in the verb \textit{exprimer} "to express", which meets the conditions. I constructed the phrase \textit{elle veut que ce produit soit parfait} "she wants this product to be perfect". We have:
\begin{enumerate} [label=(\alph*)]
  \item \textit{il veut que ce travail soit bien fait} with [\textipa{v\"o\*\#k@\*\#s@\*\#tra}]
  \item \textit{elle veut que ce produit soit parfait} with [\textipa{v\"o\*\#k@\*\#s@\*\#pro}]
\end{enumerate}
For both (a) and (b), I wanted to record the 3 basic options: no deletion, left-schwa deleted, right-schwa deleted. Since these phrases can be pronounced easily with the 2 schwas deleted,  I included this version to be recorded. I believed it would be interesting to compare between the fully deleted version, and a sentence containing the sequence as found in the language - were it should be perfectly grammatical and approved.  For this reason I added two extra phrases that contain \textit{extraordinare} and \textit{exprimer}:
\begin{itemize}
  \item \textit{c’est \textbf{extraordinaire} ce qu’il fait} "it's amazing what he does"
  \item \textit{je peux \textbf{m’exprimer} sans begayer} "I can express myself without stuttering"
\end{itemize}
Also, I maintained similarity between the phrases, in terms of length and ending sounds (\textit{fait}) - to assess the sequence [kstr]/[kspr] itself.
This is how we achieve the first 10 recordings, 5 versions for each sentence (I represent the versions with schwa deleted using an apostrophe,  as it is common in French in informal speech/writing for these cases):

\begin{center}
\begin{tabular}{||c c c||} 
 \hline
 N. & Orthography & IPA of sequence \\ [0.5ex] 
 \hline\hline
 1 & \textit{il veut que ce travail soit bien fait} &  [\textipa{v\"ok@s@tra}]  \\ 
 \hline
 2 & \textit{il veut q'ce travail soit bien fait} &  [\textipa{v\"oks@tra}]  \\ 
 \hline
 3 & \textit{il veut que c'travail soit bien fait} &  [\textipa{v\"ok@stra}]  \\ 
 \hline
 4 & \textit{il veut q'c'travail soit bien fait} &  [\textipa{v\"okstra}]  \\ 
 \hline
 5 & \textit{c’est extraordinaire ce qu’il fait} &  [\textipa{ekstra}]  \\ 
 \hline
 6 & \textit{elle veut que ce produit soit parfait} & [\textipa{v\"ok@s@pro}]  \\ 
 \hline
 7 & \textit{elle veut q'ce produit soit parfait} & [\textipa{v\"oks@pro}]  \\ 
 \hline
 8 & \textit{elle veut que c'produit soit parfait} & [\textipa{v\"ok@spro}] \\ 
 \hline
 9 & \textit{elle veut q'c'produit soit parfait} & [\textipa{v\"okspro}]  \\ 
 \hline
 10 & \textit{je peux m’exprimer sans begayer} & [\textipa{mekspri}]  \\ 
 \hline
\end{tabular}
\end{center}
% Phrases (c)-(d)
Next,  I took the two phrases Dell presented, where intuitively a version with both schwas deleted is difficult to pronounce. I slightly changed them, adding a word at the beginning of each phrase to make it a bit longer, resulting with:
\begin{enumerate} [label=(\alph*)]
  \setcounter{enumi}{2}
  \item \textit{apres, tu le retrouves} with [\textipa{t\"u\*\#l@\*\#r@\*\#truv}]
  \item \textit{j’ai envie de te battre} with [\textipa{\~avi\*\#d@\*\#t@\*\#batr}]
\end{enumerate}
Consulting with the speaker I recorded, we saw that for (c) it is possible to pronounce that version (we get the sequence [lrtr]),  although the output is a little odd. Regarding (d),  the output [dtb] is far more difficult to pronounce.  As a result, I recorded the fully deleted version for (c), but not for (d), generating 4 recordings for (c) and only 3 for (d):

\begin{center}
\begin{tabular}{||c c c||} 
 \hline
 N. & Orthography & IPA of sequence \\ [0.5ex] 
 \hline\hline
 11 & \textit{apres, tu le retrouves} &  [\textipa{t\"u\*\#l@\*\#r@\*\#truv}]  \\ 
 \hline
 12 & \textit{apres, tu l'retrouves} &  [\textipa{t\"u\*\#l\*\#r@\*\#truv}]  \\ 
 \hline
 13 & \textit{apres, tu le r'trouves} &  [\textipa{t\"u\*\#l@\*\#r\*\#truv}]  \\ 
 \hline
 14 & \textit{apres, tu l'r'trouves} &  [\textipa{t\"u\*\#l\*\#r\*\#truv}]  \\ 
 \hline
 15 & \textit{j’ai envie de te battre} &  [\textipa{\~avi\*\#d@\*\#t@\*\#batr}]  \\ 
 \hline
 16 & \textit{j’ai envie d'te battre} &  [\textipa{\~avi\*\#d\*\#t@\*\#batr}]  \\ 
 \hline
 17 & \textit{j’ai envie de t'battre} &  [\textipa{\~avi\*\#d@\*\#t\*\#batr}]  \\ 
 \hline
\end{tabular}
\end{center}
% recording
Having prepared all 17 versions, I recorded my native French speaker pronouncing all of them one-by-one. Then, I constructed a list of 15 pairs of recordings,  later to be judged by the survey's participants.  Each pair is made up by a recording marked as "A", and another one as "B". I kept the original order of the phrases I dealt with, but in each pair I decided to mix and change the order within each pair, so as to avoid some kind of expected pattern (L - deletion of left schwa, R - deletion of right schwa, D - deletion of both schwas, N - no deletion):

\begin{center}
\begin{tabular}{||c c c c||} 
 \hline
 Pair N. & A & B & Comparison\\ [0.5ex] 
 \hline\hline
 1 & 1 & 3 & N vs.  R\\ 
 \hline
 2 & 2 & 1 & L vs.  N\\ 
 \hline
 3 & 2 & 3 & L vs. R \\ 
 \hline
 4 & 4 & 5 & D vs.  valid example \\ 
 \hline
 5 & 8 & 6 & R vs. N\\ 
 \hline
 6 & 6 & 7 & N vs. L\\ 
 \hline
 7 & 6 & 8 & N vs. R\\ 
 \hline
 8 & 10 & 9 & valid example vs. D\\ 
 \hline
 9 & 11 & 13 & N vs. R\\ 
 \hline
 10 & 12 & 11 & L vs. N \\ 
 \hline
 11 & 13 & 14 & R vs. D\\ 
 \hline
 12 & 12 & 14 & L vs. D \\ 
 \hline
 13 & 17 & 15 & R vs. N \\ 
 \hline
 14 & 15 & 16 & N vs. L \\ 
 \hline
 15 & 16 & 17 & L vs. R \\ 
 \hline
\end{tabular}
\end{center}

% building 
\subsection{Experiment}
The pairs of recordings were formed as 15 questions in the online survey, where for each pair of recordings "A" and "B" there was a seperate scale: from 1 (pas d'accord) to 7 (d'accord) - that is, 1 is the lowest grade of acceptance of the pronunciation, and 7 is the highest grade. After having built the full survey,  I shared it online using the help of French speakers, who forwarded it to more people.  
Regarding the responses I received:
\begin{enumerate}
  \item A total of 22 subjects.
  \item All the subjects are native French speakers - mostly from France, but also from Belgium, Israel,  and Asia.
  \item The subjects’ ages ranged from 21 to 65 with an average of 29.22.
\end{enumerate}

\subsection{Expectations}
From conversations I had with about 3-4 French speakers regarding the OSD issue whilst preparing the experiment,  I got the impression that on the whole they agree with Dell's proposition; The default option is clearly valid. Deleting either schwa is optional and completely valid (though considered as informal speech), and the tendency to delete the left or right schwa depends on the specific French dialect, region etc. In this humble experiment I ignored these factors (which were researched for example by Bayles \& Kaplan (2016)), and focused on the general acceptance of the recordings with random French speakers. 
Regarding the fully deleted versions - most speakers agreed they sound bad, though one pointed out he thinks recording 4 is not completely wrong (he can imagine French speakers producing it, and stated he would understand the meaning with no effort). 

Therefore, I expect a general correlation with Dell's analysis: phrases 1-3,6-8,11-13,15-17 being correct. In addition, I expect some variation across phrases 4 and 9 which can be easily pronounced, and maybe not completely wrong as Dell treated them. 

\section{Conclusions} % 4 - Conclusions

\subsection{Results and analysis}
There are many ways to analyze the results of the experiment. Given the clear division into pairs using the previously defined categories (N, R, L , D) it will be interesting to see the results of the same categories across different phrases.  Also,  I'm interested in looking for the amount of consistency regrading the same pair/phrase.  Here are the main points that came up from the data:
\begin{enumerate}
  \item Though most of the answers were consistent,  I noticed two subjects with irregular answers. One (aged 21) ranked everything in the range of 6-7,  and used 5 only 3 times.  The second (aged 39) used only 1 and 7 in his answers in a pretty unique manner: phrases which got a high rank (near 7) from all other subjects, he ranked as 1,  and so as in the other way around (he ranked 7, phrases which were ranked 1 be all the rest). 
These are seemingly odd results which may come from misunderstanding of the experiment. Ignoring these two results, it seems there is a \textbf{clear consistency} within all the subjects, including within each subject.
  \item The N version (no deletion at all) given in phrases 1,5,6,10,11,15 got the average score of \textbf{6.6} as expected.
  \item The most common comparisons were between N and L/R deletion:

\begin{center}
\begin{tabular}{||c c c||} 
 \hline
 Phrases & L score & R score\\ [0.5ex] 
 \hline\hline
 1-3 & 3.5 & 2.75\\ 
 \hline
 6-8 & 3.1 & 2.97\\ 
 \hline
 11-13 & 5.3 & 3.7\\ 
 \hline
 15-17 & 3.9 & 3.7\\ 
 \hline
\end{tabular}
\end{center}
There appears to be some preference towards L (deletion of first schwa) with an average score of \textbf{3.95} compared to R (deletion of second schwa) with an average score of \textbf{3.28}.
  \item Regarding D phrases 4 and 9 (with full deletion) which were compared to a valid example in French: phrase 4 got the average score of \textbf{2.4}, and phrase 9 got the score of \textbf{2.1}.  This is clearly a low acceptance rate, however we need to examine this score compared to versions R and L which also got low scores, with the average of \textbf{3.6}. It still seems there is a preference towards a deletion of only 1 schwa over deleting both, which points out some grammaticality of deleting 1 schwa over deleting both - in accordance with Dell's proposition.
  \item D phrase 14 which had no valid example in the language, was compared to R and L versions and got the score of \textbf{1.8}. This score is lower then the other D phrases which had an average of \textbf{2.25}, however not in a significant way.
\end{enumerate}

\subsection{Possible problems and recommendations}
\begin{enumerate}
  \item During the execution of the experiment, I have been given the information that one subject found it unclear. That led me to add a clarification to the instructions: "You just have to decide if the pronunciation is a possible option in French for you".  This was added to the previous simple instruction: "...if the sentence seems correct to you". I still find the results of the experiment valid and representative. The issue of "correctness" is not so easy to explain, especially considering the fact that the experiment ignores important sociolinguistic factors that affect each individual person's schwa elision preferences.
  \item The [kstr] case: I searched for more examples and the only one I managed to build was the [kspr] example.
A more in-depth search can be done for examples similar to the [kstr] case, in which the sequence generated after deleting 2 schwas can appear as valid in French. 
  \item I chose sentences with only 2 schwas, but it will be interesting to test more complicated cases with 3 or even 4 schwas (such examples were given by Dell himself).
\end{enumerate}

\subsection{Discussion}
The OSD pattern in French is an interesting phenomenon due to its popularity and significance in literature. The generalizations given by Dell to explain the OSD logic seem to be characteristic of French speakers today. The results of this small experiment I conducted support Dell's propositions, hence they support the analysis of the VCE rule application as left-to-right. Speakers seem to have strong intuitions about this phenomenon.
My conclusions are certainly tentative because of the previously mentioned confounds in the experiment,  that should be addressed in a more extensive follow-up experiment (given my recommendations).

\clearpage
\fancyhead{} % clear all header fields
\fancyhead[RO]{\textit{REFERENCES}}

\begin{thebibliography}{} % - References
\bibitem{texbook}
Dell, François. (1973). Les règles et les sons: introduction à la phonologie générative. Paris: Hermann. Translated 1980 by Catherine Cullen as Generative phonology and French phonology. Cambridge: Cambridge University Press.
\bibitem{texbook}
Chomsky, Noam, and Morris Halle. (1968). The sound pattern of English. New York: Harper and Row.
\bibitem{texbook}
Prince, Alan, and Paul Smolensky.  (1993). Optimality Theory: Constraint interaction in generative grammar. Wiley-Blackwell: Oxford.
\bibitem{texbook}
McCarthy, John. (2007). Hidden generalizations: Phonological opacity in Optimality Theory. London: Equinox.
\bibitem{texbook}
Kiparsky, Paul. (1971). Historical linguistics. In A survey of linguistic science, ed. W. O. Dingwall, 576–642. University of Maryland Linguistics Program, College Park.
\bibitem{texbook}
Stoet, G. (2010). PsyToolkit - A software package for programming psychological experiments using Linux. 	Behavior Research Methods, 42(4), 1096-1104.  
\bibitem{texbook}
Stoet, G. (2017). PsyToolkit: A novel web-based method for running online questionnaires and reaction-time experiments. Teaching of Psychology, 44(1), 24-31.
\bibitem{corpus}
D. Goldhahn, T. Eckart \& U. Quasthoff: Building Large Monolingual Dictionaries at the Leipzig Corpora Collection: From 100 to 200 Languages.
In: Proceedings of the 8th International Language Resources and Evaluation (LREC'12), 2012
\bibitem{t}
Bayles, Andrew, Aaron Kaplan and Abby Kaplan.(2016). Inter- and intra-speaker variation in French schwa. Glossa: a journal of general linguistics 1(1): 19. 1–30, DOI: http://dx.doi.org/10.5334/gjgl.54
\end{thebibliography}

\end{document}